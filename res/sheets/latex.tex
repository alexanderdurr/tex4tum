% % % % % % % % % % % % % % % % % % % % % % % % % % % % % % % % % % % % 
% 
% LaTeX-Einführung von Emanuel Regnath				Stand: 01.01.12
%
%
%
% Allgemeine LaTeX - Hinweise:
% ======================================================================
% LaTeX Code besteht aus 3 Dingen: Text, Befehlen und Formatierungszeichen.
% Befehle haben immer die Form: \Befehl[optionaler Parameter]{Parameter}
% Formatierungszeichen sind: % \ [ ] { } # & $ 
% % kennzeichnet Kommentare
% \ leitet einen Befehl ein
% {} umklammern logisch zusammengehörige Blöcke
% die Restlichen werden später erklärt
%
% Minimales LaTeX Grundgerüst:
% \dokumentclass{scrartcl}
% \begin{document}
%	...textetxtext...
% \end{document}
%
% Tipp: am besten kompiliert man den Code und vergleicht ihn immer wieder mit der PDF-Ausgabe.
% Und Los gehts!??? Noch nicht; bevor man Text schreiben kann, muss einiges definiert werden:
%
% % % % % % % % % % % % % % % % % % % % % % % % % % % % % % % % % % % % 


% Dokumenteinstellungen
% ======================================================================

% Die Dokumentklasse definiert die Art des Dokuments
% und seine Grundeigenschaften
\documentclass[10pt,a4paper]{scrartcl}		% [Schriftgröße 10, Textbereich Din A4] {Dokumentart Artikel}


% Zusätzliche (aber sinnvolle) Pakete laden
% ======================================================================
% Pekete fügen verschiedene Funktionen zu LaTeX hinzu.
% ganz ohne Pakete wäre Latex gerade mal etwas besser als notepad...
\usepackage[a4paper]{geometry}				% DIN-A4 Größe des Papiers; sollte mit der Ausdehnung des Textes in documetnclass übereinstimmen.
\usepackage[utf8]{inputenc}					% Zeichenkodierung UTF-8 falls Probleme wegen utf8 auftreten, utf8 durch utf8x ersetzen
\usepackage[ngerman]{babel}					% Deutsche Sprache und Silbentrennung
\usepackage{amsmath}						% erlaubt mathematische Formeln
\usepackage{amssymb}						% Verschiedene Symbole
\usepackage{graphicx}						% Zum Bilder einfügen benötigt
\usepackage{hyperref}						% Sprunglinks für Überschriften, Fußnoten und Weblinks



% Eigene Befehle und Befehlsanpassungen (optional)
% ======================================================================
% Form: \newcommand{\neuerBefehl}[Anzahl Parameter]{ Befehlsreihe oder Text #n }   n = 1,2,...
% 		#n repräsentiert den übergebenen Parameter
\newcommand{\norm}[1]{ \| #1 \| }									%Vektornorm
\newcommand{\chem}[1]{\ensuremath{\displaystyle{ \mathrm{#1}}}}		%chemische Formeln die gut aussehen ;)

% Überschreibungen
\renewcommand{\arraystretch}{1.4}				% Vergrößert Abstände in Tabellen


% Dokumentbeschreibung
% ======================================================================
\title{LaTeX-Einführung für Artikel}
\author{Emanuel Regnath}




 


% Dokumentbeginn
% ======================================================================
% Ab hier beginnt das eigentliche Dokument.
% Alles was danach folgt wird im fertigen PDF angezeigt.

\begin{document}


% Titel		(Titel, Autor, Erstellungsdatum)
% ===========================================
\maketitle

% Vertikaler Abstand
% ===========================================
\vspace{4.0cm}		% Gültige Einheiten für Längenangaben: pt, px, mm, cm, em

% Inhaltsverzeichnis   (Überschriften werden automatisch in das Inhaltsverzeichnis aufgenommen.)
% ===========================================
\tableofcontents			


% Neue Seite
% ===========================================
\newpage



% -------------------------------------------
% | 		E I N L E I T U N G				|		(Kommentare zur Übersicht sind wichtig!)
% ~~~~~~~~~~~~~~~~~~~~~~~~~~~~~~~~~~~~~~~~~~~
% ======================================================================================================
\section{Einleitung}	% \section{Überschrift}

In Latex kann Text einfach geschrieben werden. Auch Umlaute wie ä, ö, ü oder ß sind mit dem UTF-8 package kein Problem.
	Einrückung, viele           Leerzeichen oder Absätze im Editor 
haben keine Auswirkung auf den Textfluss.
Erst ein Doppelbackslash\\											% oder \newline aber \\ ist kürzer ;)
erzeugt einen Zeilenumbruch. Ein doppelter Absatz im Editor			% oder \par aber doppel-Enter ist kürzer :)

bewirkt einen eingerückten Absatz im Latex-PDF.
Will man eine Leerzeile erreichen, bedient man sich folgender Methode:\\
\\
Nun ist eine Leerzeile entstanden ohne Einrückung.
Verschiedene Einrückungen oder Abstände erreicht man so\,so\;so\ so\quad so\qquad so \hspace{2.0cm}so. Negativen Abstand s\!o s\!\!o bzw. \ s\!\!\!\!\!o \par
Da folgende Zeichen Befehle oder Formatierungszeichen darstellen, müssen sie im Text etwas aufwendiger geschrieben werden: \textbackslash  \$ \{ \} \& \% \#


% -------------------------------------------
% | 		B E F E H L E					|
% ~~~~~~~~~~~~~~~~~~~~~~~~~~~~~~~~~~~~~~~~~~~
% ======================================================================================================
\section{Befehle}
... beginnen in \LaTeX{} \emph{immer} mit einem Backslash. \\		% Form: \Befehl[optionaler Parameter]{Parameter}
Textformatierungen: normal, \underline{unterstrichen}, \textbf{fett}, \textit{kursiv}, \textsl{schräg} ,\textsc{Großbuchstaben} \\
Drei Schriftarten: \textnormal{normal}, \textsf{seriflos}, \texttt{typewriter}\\	
% Das gleiche bewirken \rmfamily, \sffamily und \ttfamily, allerdings bis zum Ende eines Blockes{} oder bis \normalfont
{Geschweifte \sffamily Klammern { } kennzeichen {\ttfamily logische Blöcke}, werden aber nicht gerendert.} \\[5pt]  					%Neue Zeile mit zusätzlichem Abstand
Fußnoten\footnote{Ich bin eine Fußnote} sind total einfach\footnote{Automatische Nummerierung und Formatierung}.
Es gibt drei Arten von Bindestrichen: - -- ---
	

% -------------------------------------------
% | 		G L I E D E R U N G				|
% ~~~~~~~~~~~~~~~~~~~~~~~~~~~~~~~~~~~~~~~~~~~
% ======================================================================================================

\section{Gliederung: Mächtige Überschrift}
% ======================================================================================================

	\subsection{Unterüberschrift}
	% ------------------------------------------------------------------
	
		\subsubsection{Kleinlaute Überschrift}
		
			\paragraph{Paragraph 1}	%  Wird nicht ins Inhaltsverzeichnis aufgenommen
			Text bla bla bla. Weiter Gliederungsmöglichkeiten sind Listen. Auf alle Fälle sollte \textbf{Fettschrift} nicht zur Gliederung verwendet werden.
			% Anmerkung in der Dokumentclass "book" gibt es über \section noch das \chapter


% -------------------------------------------
% | 		U M G E B U N G E N				|
% ~~~~~~~~~~~~~~~~~~~~~~~~~~~~~~~~~~~~~~~~~~~
% ======================================================================================================
\section{Umgebungen}
Es gibt verschiedene Umgebungen in Latex, die sich vom normalen Text und Befehlen abgrenzen. 
Zum Beispiel Tabellen, Formeln oder Zitate. Sie werden mit ``begin\{ Umgebung \}'' und ``end\{ Umgebung \}'' gekennzeichnet. 
Zum Beispiel die quote-Umgegbung:
\begin{quote}
	\glqq Ich weiß, dass ich nichts weiß.\grqq \ --\ \textsc{Sokrates}				%Anführungszeichen deutsch: \glqq,\grqq  englisch: ``,''
\end{quote}
Auch wenn man diese Formatierung anders ereichen könnte, 
ist es sinnvoll und guter Stil ein Zitat in eine quote-Umgebung zu packen, 
da man eigene Formatierungen für Umgebungen festlegen kann und so alle Zitate auf einmal richtig formatiert werden.


	\subsection{Tabellen}
	% ------------------------------------------------------------------
	sind auch leicht:\\
	\begin{tabular}{l||lc|r}				%{lclr} gibt die Anzahl(4) und die Ausrichtung der Spalten an. | bedeutet eine vertikale Trennlinie. 
		links 			& links & zentriert 	& rechts\\		% ein & trennt Spalten \\ beginnt eine neue Zeile
		etwas Mathe 	& $3 \cdot 5 + \frac{1}{2}$ & $\ne 0$ & $\Rightarrow$ Ich weiß doch was\\ \hline 	% \hline erzeugt eine horizontale Linie
		leere Spalten 	& & & \\
	\end{tabular}


	\subsection{Formeln}
	% ------------------------------------------------------------------
	Kleine mathematische Ausdrücke wie $f(x) = a \cdot x^2 + 2x$, chemische Formeln $H_3O^+$ oder Symbole $\Rightarrow$, können in den Textfluss 
	mit Dollarzeichen eingebunden werden, größere Formeln werden mit der equation-Umgebung abgesetzt und nummeriert:
	\begin{equation}
		\lambda := \lim\limits_{x_1 \rightarrow \infty}    \int\limits_{x_0}^{x_1} \frac{ f(t) }{ \sqrt{t^2 + \sin^2(t) } } \; \mathrm dt \stackrel{!}\ge 1
		% Diese Formel dient lediglich als Beispiel wie man einige mathematische Ausdrücke am besten in LaTeX umsetzt	
	\end{equation}
	
	Will man in Formeln Text einfügen, muss man diesen extra kennzeichen:\\
	$\text{falls } v \in \mathbb R^n,\ \lambda \in \mathbb R \text{ so gilt } \forall v \exists \lambda: \norm{ \lambda \cdot \vec v } = 1, denn so gehts nicht$\\
	\\
	Innerhalb von Formeln gibt es wiederum eigene Umgebungen wie Arrays oder Matrizen. Sie werden wie Tabellen formatiert:
	\begin{equation}
		\text{Matrix: } A^\top = \begin{bmatrix} a_{11} & \hdots & a_{1n} \\ \vdots & \ddots & \vdots \\ a_{m1} & \hdots & a_{mn} \end{bmatrix} \in \mathbb C^{m \times n} \qquad 
		\text{Vektor: } \vec v = \begin{pmatrix} v_1 \\ v_2 \\ v_3 \end{pmatrix}
	\end{equation}
	\\
	Man kann Formeln auch Einrahmen: \boxed{ f:D \subseteq \mathbb R^n \rightarrow \mathbb R^m ,\, x \mapsto f(x) } \\
	Positionierung in Formeln: $m^{hoch} \quad m_{tief} \quad \overset{oben}{mitte} \quad \underset{unten}{mitte}$ \qquad speziell: $\xrightarrow{a \ne 0}$ und $\stackrel{!}\le$\\
	% Eine vollständige Liste von mathematischen Ausdrücken findet man im Internet.
		\subsubsection{Beispielformeln}
		Physik: $v = \frac{\partial s}{\partial t} = 3.14 \mathrm{\frac{m}{ s}}$\\	
		Chemische Formeln: $\chem{2H_2O \rightleftharpoons H_3 O^{+} + OH^-}$\\
		Strahlungsphysik: $\chem{ ^{14}_{\ 6} C \rightarrow {}^{14}_{\ 7} N + e^- + \overline \nu_e}$


	\subsection{Listen}
	% ------------------------------------------------------------------
	Es gibt drei Arten von Listen(itemize, enumerate, description) die mit der Erklärung über guten LaTeX Stil veranschaulicht werden.
	Damit ein LaTeX Dokument	% Mit \itemsep4pt nach \begin{...} kann man den Abstand anpassen
	\begin{itemize}\itemsep-2pt	
		\item übersichtlich bleibt,
		\item für sich und Andere später noch nachvollziehbar ist
		\item und eventuelle Nachbesserungen einfach sind,
	\end{itemize} 
	sollte man ein paar Regeln beachten:
	\begin{enumerate}			
		\item Einrücken und Kommentare im Quellcode
		\item Dokumentweite Einstellungen und eigene Befehle immer vor \textbackslash begin\{document\} schreiben
		\item Befehle nicht zweckentfremden\\ 
			$\Rightarrow$ Die von LaTeX vorgesehenen Befehle anpassen oder eigene erstellen.			%Es gibt für alles den richtigen LaTeX-Befehl, man muss ihn nur finden		
	\end{enumerate}
	
	Zum Beispiel bewirken \emph{Hervorhebung} und \textit{kursiv} zunächst das selbe. 
	Will man aber Hervorhebungen fett geschrieben haben so überschreibt man einfach den Befehl \textbackslash emph:
	\renewcommand{\emph}[1]{\textbf{#1}}	%\renewcommand{\Befehl}[Anzahl der Parameter]{ Neuer Befehl {#Parameternummer} }
	Somit bleiben \textbf{fett} und \textit{kursiv} unverändert und \emph{Hervorhebungen} sind flexibel anpassbar.\\
	% Will man einen Befehl mit sich selbst überschreiben, muss man ihn zuerst mit \let kopieren.
	% Beispiel: Vektoren fett schreiben:
	% \let\oldvec = \vec 	   \renewcommand{\vec}[1]{\oldvec{\boldsymbol{#1}}}
	\subsubsection{Querverweise} werden durch labels gesetzt.\label{hier} Es gibt drei Arten von Verweisen:
	\begin{description}	
		\item[Verweis] mit ref \ref{hier}
		\item[Seitenverweis] wird mittels pageref \pageref{hier} eingefügt
		\item[Literaturverweise:] Eine Tatsache aus einem Buch \cite{bowie}
	\end{description}

	
% -------------------------------------------
% | 		B I L D E R						|
% ~~~~~~~~~~~~~~~~~~~~~~~~~~~~~~~~~~~~~~~~~~~
% ======================================================================================================	
\section{Bilder}
Bilder können mit dem Befehl \textbackslash includegraphics[width = ?, height = ?]\{ Pfad zum Bild.pdf\} eingebunden werden.		% Beispiel: \includegraphics[height = 3.0cm]{./vektor.pdf}
Allerdings unterstützt LaTeX nur wenig Formate: .pdf .jpg .png .gif.
Nur .pdf kann Vektorgrafiken enthalen. 		% Vektorgrafiken sind leichter zu erstellen, verlustfrei, und klein. :)



% .:: Wie Latex funktioniert
% ======================================================================================================
\section{Wie LaTeX funktioniert}
Grob gesagt macht \LaTeX{} um jedes Zeichen eine Box: \fbox{H}, dann um jedes Wort: \fbox{ \fbox{H}\fbox{a}\fbox{l}\fbox{l}\fbox{o} }, dann um jede Zeile, usw.
Will man große Objekte und einen Text nebeneinander haben, muss man um beides eine Box bauen, denn in einer Zeile dürfen nur Boxen nebeneinander aber niemals übereinander stehen.
Dies erreicht man am besten mit parbox oder pbox:\\
\parbox{3.0cm}{
	\begin{tabular}{l||l|l}
	a & b & c \\ \hline \hline
	1 & 2 & 4 \\
	4 & 5 & 6 \\
	\end{tabular}
} 
\parbox{10cm}{So hier ist ein ganz normaler Text, der allerdings über mehrere Zeilen neben der Tabelle erscheint. Text und Tabelle sind sozusagen zwei \glqq große\grqq\ Wörter innerhalb einer Zeile.}



% .:: Literaturverzeichnis
% ======================================================================================================
\begin{thebibliography}{------}
\bibitem[Bowie87]{bowie}		%[name]{label}
	J. U. Bowie, R. L\"uthy and D. Eisenberg.
	{\em A Method to Identify Protein Sequences That Fold 
	into a Known Three-Dimensional Structure}.
	Science, 1991 (253), pp 164-170
\end{thebibliography}



% Geschafft!
% Nun liegt es an dir, dein eigenes LaTeX Dokument zu erstellen





% Vorsicht!!! hier endet das Dokument.
% Der Rand der sichtbaren Welt, danach hat nichts mehr etwas zu suchen.
% ==============================================================================================================
\end{document}













haaaallo? ließt mich jemand?
ich bin ein kleiner verirrter Text, der nirgends zu sehen ist.
Was soll ich jetzt bloß machen? Ich will zurück zu meiner Mama!!! *schluchtz*

